\documentclass[
fontsize=11pt,
paper=a4,
numbers=noenddot,
parskip=full
%draft
]{scrartcl}

\usepackage[utf8]{inputenc}
\usepackage[T1]{fontenc}

\usepackage[backend=biber,style=alphabetic,maxcitenames=2]{biblatex}
%\addbibresource{bibliography.bib}

\include{preambl}e
\def\R{\mathbb{R}} 
\def\N{\mathbb{N}}
\def\Z{\mathbb{Z}}
\def\K{\mathbb{K}}
\def\C{\mathbb{C}}
\def\Rp{\mathbb{R}_{+}}
\def\F{\mathbb{F}}
\def\I{\mathbb{I}}

\def\iff{~\Leftrightarrow~}

\def\eps{\mathtt{eps}}
% function spaces
\def\CC{\mathscr{C}}
% space of polynomials
\def\P{{\mathcal{P}}}

% integrals, derivatives
\def\d{\mathrm{d}}
\def\dx{\,\mathrm{d}x}
\def\dy{\,\mathrm{d}y}
\def\dt{\,\mathrm{d}t} 

\def\rd{\text{rd}}

\def\Df{\mathrm{D}f}

\def\O{\mathcal{O}}

\def\rq{\mathrm{R}}

% duality pairing
\def\<{\langle}
\def\>{\rangle}

% Swap the definition of \abs* and \norm*, so that \abs
% and \norm resizes the size of the brackets, and the 
% starred version does not.
\DeclarePairedDelimiter\abs{\lvert}{\rvert}%
\DeclarePairedDelimiter\norm{\lVert}{\rVert}%

\makeatletter
\let\oldabs\abs%
\def\abs{\@ifstar{\oldabs}{\oldabs*}}
%
\let\oldnorm\norm%
\def\norm{\@ifstar{\oldnorm}{\oldnorm*}}
\makeatother

\makeatletter
\newcommand*{\tp}{%
  {\mathpalette\@transpose{}}%
}
\newcommand*{\@transpose}[2]{%
  % #1: math style
  % #2: unused
  \raisebox{\depth}{$\m@th#1\intercal$}%
}
\makeatother

\makeatletter
\newcommand*{\herm}{%
  {\mathpalette\@hermitian{}}%
}
\newcommand*{\@hermitian}[2]{%
  % #1: math style
  % #2: unused
  \raisebox{\depth}{$\m@th#1{\mathrm{H}}$}%
}
\makeatother

% vertical equals 
\newcommand{\verteq}{\rotatebox{90}{$\,=$}}
% underset with vertical equals
\newcommand{\equalto}[2]{\underset{\scriptstyle\overset{\mkern4mu\verteq}{#2}}{#1}}

% for fractions with bigger elements or nested fractions
\newcommand{\ffrac}[2]{\ensuremath{\frac{\displaystyle #1}{\displaystyle #2}}}

\DeclareMathOperator{\rang}{rang}
\DeclareMathOperator{\cond}{cond}
\DeclareMathOperator{\diag}{diag}
\DeclareMathOperator{\im}{im}
\DeclareMathOperator{\spr}{spr}

\DeclareMathOperator*{\argmax}{arg\,max}
\DeclareMathOperator*{\argmin}{arg\,min}

\DeclarePairedDelimiterX{\inp}[2]{\langle}{\rangle}{#1, #2}

\newcommand*{\mat}[1]{\bm{#1}}
\let\oldvec=\vec
\renewcommand{\vec}[1]{\mathbf{#1}}

%%%%%%%%%%%%%%%%%%%%%%%%%%%%%%%%%%%%%%%%%%%%%%%%
%%%%% Theorem environments %%%%%%%%%%%%%%%%%%%%%
\newtheorem{theorem}{Theorem}
\newtheorem{corollary}[theorem]{Corollary}
\newtheorem{lemma}[theorem]{Lemma}
\newtheorem{proposition}[theorem]{Proposition}

\theoremstyle{definition}
\newtheorem{definition}[theorem]{Definition}
\newtheorem{example}[theorem]{Example}
\newtheorem*{remark*}{Remark}
\newtheorem{remark}[theorem]{Remark} 
 
%%% Local Variables:
%%% mode: latex
%%% TeX-master: "main"
%%% End:
 

% DRAFT PACKAGES
\usepackage{showlabels}

\begin{document}
\section*{Basics of Monte Carlo Methods}
Before discussing Sequential Monte Carlo methods and Particle Filters
in more detail, we introduce the basics of Monte Carlo methods that
will be needed later.

In this section we consider approximating for a fixed $n \in \N$ the
probability density $\pi_n(x_{1 : n})$ where
$x_{1:n} \coloneqq (x_1, \dotsc, x_n)$. A Monte Carlo method
approximates $\pi_n(x_{1:n})$ by the empirical measure
\[
  \hat{\pi}_n(x_{1:n}) = \frac{1}{N} \sum_{i = 1}^N
  \delta_{X_{1:n}^i}(x_{1:n})\,,
\]
where the $X_{1:n}^i$, $i = 1, \dotsc, N$ are $N$ independent samples
of $\pi_n(x_{1:n})$ and $\delta_{x_0}(x)$ denotes the Dirac delta mass
located at $x_0$. The expectation of a test function
$\phi_n : \mathcal{X}^n \rightarrow \R$ given by
\[
  \mathbb{E}_n(\phi_n) \coloneqq \int \phi_n(x_{1:n})
  \hat{\pi}_n(x_{1:n}) dx_{1:n}
\]
can then be estimated by
\[
  \mathbb{E}_n^{\text{MC}}(\phi_n) \coloneqq \int \phi_n(x_{1:n})
  \hat{\pi}_n(x_{1:n}) dx_{1:n} = \frac{1}{N} \sum_{i=1}^N
  \phi_n(X_{1:n}^i) \,.
\]
It is straightforward to check that this estimator is unbiased and
that the variance of the approximation error decreases
\textbf{independent of the dimension} of the space $\mathcal{X}^n$ at
a rate of $\mathcal{O}(1/N)$. This is usually not the case with
traditional numerical integration methods where an increase in the
dimension of the integral makes its approximation considerably harder.

Note, however, that this Monte Carlo approach requires sampling from
$\pi_n(\cdot)$ which might not be possible for complex
high-dimensional distributions. To this end, we introduce a
\textbf{proposal density} $q_n(x_{1:n})$ that is only required to be
defined on the same support as $\pi_n(x_{1:n})$. In a Bayesian
framwork writing
\[
  \pi_n(x_{1:n}) = \frac{\gamma_n(x_{1:n})}{Z_n} =
  \frac{\gamma_n(x_{1:n})}{\int \gamma_n(x_{1:n}) dx_{1:n}}
\]
we require only that
$\gamma_n : \mathcal{X}^n \rightarrow (0, \infty)$ is known pointwise,
whereas the \emph{normalising constant} $Z_n$ might be unknown. With
the proposal density this can be rewritten as
\begin{equation}
  \label{eq:is:density}
  \pi_n(x_{1:n}) = \frac{w_n(x_{1:n})q_n(x_{1:n})}{Z_n} = \frac{w_n(x_{1:n})q_n(x_{1:n})}{\int w_n(x_{1:n})q_n(x_{1:n})dx_{1:n}}
\end{equation}
where $w_n(x_{1:n})$ is the \textbf{unnormalised weight} function
\[
  w_n(x_{1:n}) = \frac{\gamma_n(x_{1:n})}{q_n(x_{1:n})}\,.
\]
Assume we draw $N$ independent samples $X_{1:n}^i \sim q_n(x_{1:n})$.
We can then estimate $\pi_n(x_{1:n})$ by
\[
  \hat{\pi}_n(x_{1:n}) = 
\]
\end{document}

 