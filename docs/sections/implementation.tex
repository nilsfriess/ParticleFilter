\section*{Implementation}
In this section we present an implementation of a particle filter in
C++. The particle filter itself is implemented as a
dependency-free\footnote{The library can be configured to run some
  parts in parallel. In that case the program has to be linked against
  Intel's Threading Building Blocks (TBB) library~\cite{intel}. If the
  parallel capabilities are not used, the library depends only on the
  C++ standard library.} header-only generic library that, while being
easy to set up and use, is versatile and can be used with a wide
variety of problems. This is demonstrated by three examples, two of
which are based on the same problem but are using different prior and
proposal distributions.

The code can be found at
\url{github.com/nilsfriess/ParticleFilter}. The GitHub repository
contains a CMake~\cite{cmake} project containing the actual library in
the folder \texttt{libs/smcpf} and three examples located in the
folder \texttt{apps}. Information about the dependencies of the
individual examples and instructions on how to build and run them can
be found in the \texttt{README.md} file inside the root folder of the
repository. The \texttt{data} folder contains sample data to use with
the examples and scripts that were used to generate the sample data.

The library consists of the following classes (and their respective
header files)
\begin{itemize}
\item Particle \texttt{(particle.hh)}
\item ParticleFilter \texttt{(particlefilter.hh)}
\item Model \texttt{(model.hh)}
\item History \texttt{(history.hh)}
\end{itemize}
All of these classes lie in a namespace \texttt{smcpf} and are
templated to allow for arbitrary particle types, e.\,g.\ the
\texttt{Particle} class, that holds the value and weight of a single
particle is of the following form
\begin{minted}{cpp}
template <class PT> 
class Particle {
private:
  PT m_value;
  double m_weight;
...
}; 
\end{minted}
where the particle type \texttt{PT} could take values of some finite
set, be a real number (\ie, a \texttt{double}) or a $n$-dimensional
vector etc. The \texttt{ParticleFilter} class implements the
algorithms introduced above. It takes the following template
parameters
\begin{minted}{cpp}
template <class PT, class OT, 
          size_t N,  
          typename... Args>
class ParticleFilter { ... }

\end{minted}
where \texttt{PT} and \texttt{OT} denote the type of particle and
observation, respectively and \texttt{N} is the number of
particles. The last template parameter \texttt{Args} is
\emph{parameter pack} that can hold an arbitrary number of additional
arguments of any type. The use for such arguments is explained in the
description of the \texttt{Model} class below.

The library can be configure to run parts of it in parallel, \eg the
particle evolution (see below). To enable this feature, one has to
define the prepocessor constant \texttt{PF\_USE\_PARALLEL} before the
header file \texttt{particlefilter.hh} is included, \eg by
\begin{minted}{cpp}
#define PF_USE_PARALLEL
#include <particlefilter.hh>
\end{minted}

Internally, this uses the C++17 \texttt{execution} header from the C++
standard library that depends on Intel's TBB
library~\cite{intel}. This means, if \texttt{PF\_USE\_PARALLEL} is
set, the TBB header files have to be available for the compiler and
the program has to be linked against the library \texttt{tbb}. Not
using the parallel capabilities of the library does neither change the
way it has to be used nor does it limit its capabilities. It only
affects how the internal algorithms are run.

Apart from these compile-time parameters, to construct a
\texttt{ParticleFilter} one also needs to provide an instance of a
\texttt{Model}, a resampling strategy\footnote{At the moment, only
  systematic resampling is implemented.}, a resampling threshold and
an initial seed for the random number generator (rng). Additionally, a
boolean parameter that specifies whether some information from every
time step should be held in memory has to be given (\eg for debugging
or plotting, see examples below). Only the first parameter is
mandatory, i.\,e.\ the signature of the constructor of the
\texttt{ParticleFilter} class is given by
\begin{minted}{cpp}
ParticleFilter(
    Model<PT, OT, Args...> *t_model,
    bool t_enable_history = false,
    ResamplingStrategy t_strategy = ResamplingStrategy::RESAMPLING_SYSTEMATIC,
    double t_treshhold = 0.5, 
    double t_seed = 0)
\end{minted}
The value of \texttt{t\_strategy} can either be the default
\texttt{RESAMPLING\_SYSTEMATIC} or \texttt{RESAMPLING\_NONE}, both of
which are defined in the enumeration \texttt{ResamplingStrategy}. The
value of \texttt{t\_threshold} is used to decide when to actually
perform resampling. At each time step an estimate of the effective
sampling size is computed using~\eqref{eq:ESS}. The particles are only
resampled if this value is below \texttt{t\_threshold} $\times$
\texttt{N}. Thus, \texttt{t\_threshold} should take values between
zero and one (since ESS takes values between 1 and \texttt{N}), where
a value of zero implies that the particles are never resampled and one
leads to resampling being performed at each step. Before explaining
the methods defined inside the \texttt{ParticleFilter} we discuss the
\texttt{Model} class.

This class is implemented as an abstract base class (sometimes called
\emph{interface}), meaning that the class itself cannot be
instantiated. Therefore, in order to define a model, a class that is
derived from \texttt{Model} has to be implemented. The class is also
templated with the following parameters
\begin{minted}{cpp}
template <class PT, 
          class OT, 
          typename... Args> 
class Model { ... }
\end{minted}
where \texttt{PT} and \texttt{OT} are again the particle and
observation type. To explain the usage of the parameter pack
\texttt{Args} we first discuss the virtual functions of the class:
\begin{minted}{cpp}
virtual PT zero_particle() = 0;

virtual void sample_prior(Particle<PT> &t_particle) = 0;

virtual double update_weight(const Particle<PT> &t_particle_before_sampling,
                             const Particle<PT> &t_particle_after_sampling,
                             const OT &t_observation, 
                             Args... t_args) = 0;

virtual PT sample_proposal(const Particle<PT> &t_particle,
                           const OT &t_observation, 
                           Args... t_args) = 0;
\end{minted}
All these methods are \emph{pure virtual} methods meaning that a model
class that derives from this class must implement all four of
them. They all get automatically called by the
\texttt{ParticleFilter}. The first method should return a value of
type \texttt{PT} that represents zero, \ie, in the one dimensional
case where \texttt{PT = double} this should be 0, if \texttt{PT} is
\eg two dimensional, this method should return the 2D zero vector and
analogously for higher dimensions and other particle types. The method
\texttt{sample\_prior} is used to initialise the set of particles. It
has to set the value of the given particle \texttt{t\_particle} using
the \texttt{set\_value} method of the particle. This is different from
the \texttt{update\_weight} and \texttt{sample\_proposal} methods that
\emph{do not} alter the particle themselves. Rather they should return
the value the the particle's weight should get multiplied by and the
particle's new value, respectively (see examples below).

The first parameter of the \texttt{update\_weight} method is the
particle before \texttt{sample\_proposal} is called and the second
after it is called. This is useful, since the developer cannot specify
the order in which these two methods are called. However, some models
require the value of the particle before it has been updated (cf.\
Example~\ref{ex:lv1}) and some after the sampling step (cf.\
Example~\ref{ex:1} and~\ref{ex:lv2}). The type of the last parameter
\texttt{t\_args} of both methods is specified by the template
parameter pack \texttt{Args}. It can be used to supply an arbitrary
number of additional parameters of arbitrary types to these methods.
This can be used if the proposal pdf and weight update function depend
on additional parameters like the current time step, which is the case
in all of the following examples. The types provided as \texttt{Args}
to the \texttt{Model} class and those provided to the
\texttt{ParticleFilter} must match, otherwise the program will not
compile. The following simple example is used to demonstrate how a
model can be defined.

\begin{example}\label{ex:1}
  This example has been studied in a number of publications before,
  see for example~\cite{arulampalam,gordon,kitagawa}. The
  implementation can be found in the file
  \texttt{apps/example1/main.cc}. Let
  \begin{align*}
    p(x_0) &= \mathcal{N}(0,0.5)\\
    p(x_n \mid x_{n-1}) &= \mathcal{N}(x_n;\, h_n(x_{n-1};n), \sigma_{\text{sys}})\\
    p(y_n \mid x_n) &= \mathcal{N}(y_n;\, \frac{{x_n}^2}{20},
                      \sigma_{\text{obs}})
  \end{align*}
  where
  \begin{equation}
    \label{eq:ex1:h}
    h_n(x_{n-1};n) = \frac{1}{2}x_{n-1} + \frac{25 x_{n-1}}{1 + x_{n-1}^2} + 8 \cos(1.2n) \\
  \end{equation}
  and $\mathcal{N}(\mu, \sigma)$ denotes a Gaussian distribution with
  mean $\mu$ and variance $\sigma$ and $\mathcal{N}(x; \mu, \sigma)$
  denotes a Gaussian pdf with mean $\mu$ and variance $\sigma$
  evaluated at $x$. We choose $\sigma_{\text{sys}} = 10$ and
  $\sigma_{\text{obs}} = 1$.

  Since both the particles and observations are one-dimensional we set
  \texttt{PT = double,\ OT = double}. Note that $h$ defined
  in~\eqref{eq:ex1:h} depends on the index of the current time step
  $n$. Therefore, we provide one additional template parameter, \ie,
  we set \texttt{Args = int}. Hence, the model could be defined as
\begin{minted}{cpp}
class ExampleModel : public Model<double, double, int> {
\end{minted}
  Here, the zero particle is simple the value \texttt{0.0} and the
  prior is a Gaussian, centred at 0 with variance 0.5.
\begin{minted}{cpp}
...
public:
  virtual double zero_particle() override { return 0.0; }

  virtual void sample_prior(Particle<double> &t_particle) override {
    t_particle.set_value(m_prior(m_gen));
  }
\end{minted}
  where \texttt{m\_gen} and \texttt{m\_prior} are defined as
\begin{minted}{cpp}
...
private:
  std::mt19937 m_gen;
  std::normal_distribution<> m_prior{0, 0.5};
\end{minted}
  The class \texttt{std::mt19937} defines a pseudo rng based on the
  \emph{Mersenne Twister} algorithm.

  To implement the remaining two methods we first have to choose a
  proposal density. For simplicity, we use the prior
  $p(x_n \mid x_{n-1})$, \ie, we implement a bootstrap particle
  filter.  The recursive weight update
  formula~\eqref{eq:weight_update} then simplifies to
  \[
    w_k^{(i)} \propto w^{(i)}_{k-1} p(y_k \mid x_k^{(i)}) \,,
  \]
  and we obtain
\begin{minted}{cpp}
  virtual double
  update_weight(const Particle<double> & /*t_particle_before_sampling*/,
                const Particle<double> &t_particle_after_sampling,
                const double &t_observation, int /*t_step*/) override {
    auto pval = t_particle_after_sampling.get_value();
    auto mean = (pval * pval) / 20.0;
    auto var = 1.0;

    return normal_pdf(t_observation, mean, var);
  }
\end{minted}
  The function \texttt{normal\_pdf(double, double, double)} is defined
  in the file \texttt{helper.hh} and simply implements a Gaussian pdf.
  Note that the \texttt{update\_weight} method does not return the new
  value of the weight but rather the value the weight should be
  multiplied with, \ie, in our case $p(y_k \mid x_k^{(i)})$.

  Since in this case the proposal is the prior, the
  \texttt{sample\_proposal} method is a straightforward implementation
  of $p(x_k^{(i)} \mid x_{k-1}^{(i)})$ defined by the model above.
\begin{minted}{cpp}
  virtual double sample_proposal(const Particle<double> &t_particle,
                                 const double & /*t_observation*/,
                                 int t_step) override {
    auto pval = t_particle.get_value();
    auto mean = pval / 2.0 + (25 * pval) / (1 + pval * pval) +
                8 * std::cos(1.2 * (t_step));
    auto var = 10.0;

    std::normal_distribution<> proposal{mean, var};
    return proposal(m_gen);
  }
\end{minted}
  With these four methods the model is fully defined and can be used
  to construct a \texttt{ParticleFilter}. In the file
  \texttt{apps/example1/main.cc} two additional methods
  \begin{itemize}
  \item \mint{cpp}|void load_observations(const std::string
    &t_filename)|
  \item \mint{cpp}|std::optional<std::pair<double, double>>
    next_observation()|
  \end{itemize}
  are defined. The first method loads observations from a \texttt{csv}
  file (comma-separated values) and the second method returns a new
  observation each time it is called until all observations have been
  read. An observation consists of a time index and the actual
  observation. The artificial observations can be generated using the
  Python script \texttt{data/example1/gen\_obs\_ex1.py} that generates
  100 random samples $x_k \sim p(x_k \mid x_{k-1})$ where
  $x_0 \sim p(x_0)$. The observations are then generated by squaring
  these values, dividing them by 20 and perturbing them by additive
  Gaussian noise such that they are distributed according to
  $p(y_k \mid x_k)$. These artificial observations and the
  corresponding time steps are then written to a file that can be read
  by the method \texttt{load\_observations}.

  With this model we con now easily implement the actual particle
  filter:
\begin{minted}{cpp}
int main() {
  constexpr size_t N = 400;

  typedef smcpf::ParticleFilter <double, // Type of particle
                                 double, // Type of observation
                                 N,      // Number of particles
                                 int>    // index of observation
                                 PF;
  ExampleModel model;

  PF pf(&model, true);
  ...
\end{minted}
  The observations are automatically loaded inside the constructor of
  the model and the initial set of particles is initialised inside the
  constructor of the \texttt{ParticleFilter}. To run the actual
  filtering we have to call the method \texttt{evolve}, defined as
\begin{minted}{cpp}
void ParticleFilter::evolve(OT t_observation,  Args... t_args)
\end{minted}
  This method calls the \texttt{sample\_proposal} and
  \texttt{update\_weight} method of the model for each of the
  particles. If the template parameter \texttt{parallel} is set to
  true, this is done in parallel. This requires no extra care since
  the particles are all independent and altering the value or weight
  of one particles does not affect any of the other particles.  In
  addition, this method checks whether resampling is necessary and
  possibly resamples the particles. Resampling, however, cannot be
  easily parallelized (there are some publications discussing the
  parallelization of resampling algorithms, see \eg~\cite{murray}).

  The \texttt{evolve} method has to be called at each time step. Using
  the \texttt{next\_observation} method of our model we do this as
  follows
\begin{minted}{cpp}
  while (auto obs = model.next_observation()) {
    pf.evolve(obs.value().second, obs.value().first);
  }
\end{minted}
  Using the method
\begin{minted}{cpp}
PT ParticleFilter::weighted_mean()
\end{minted}
  one can obtain a Monte Carlo estimate of the expectation of a random
  variable with pdf $p(x_k \mid y_{1:k})$ at each time step $k$. Since
  we set the \texttt{history} flag to \texttt{true}, these means are
  all held in memory and we can read them after all observations have
  been processed. To that end, the \texttt{ParticleFilter} class
  provides a method
\begin{minted}{cpp}
  template <class PTWriter, class ArgWriter>
  void write_history(std::ostream &t_out, 
                     PTWriter t_writer,
                     ArgWriter t_awriter, 
                     char t_separator = ',')
\end{minted}
  that writes the contents of the history to the output stream
  \texttt{t\_out} (\eg a file stream) using the functors
  \texttt{PTWriter} and \texttt{ArgWriter}, which have to be
  implemented such that they convert values of types \texttt{PT} and
  \texttt{Args...} to \texttt{std::strings} (\ie, they define how to
  print these values). For a detailed description on the usage of this
  method see the example files and the comments in the definition of
  the particle filter class.

  Figure~\ref{fig:ex1} shows the plot of one exemplar run together
  with the generated observations and simulated values using the model
  and particle filter described above.
  
\end{example}
The following two examples are based on the same problem and
demonstrate how third-party libraries can be used to define arbitrary
particle types.

\begin{example}[Lotka Volterra using bootstrap filter]\label{ex:lv1}
  Given approximations of the population size of a certain species
  (the \emph{predators}) we want to estimate the quantity of another
  species (the \emph{preys}) and estimate their respective population
  sizes. The predators and preys are assumed to follow the so called
  \emph{Lotka-Volterra model}. This model is mainly described by the
  following system of differential equations:
  \begin{equation}
    \label{eq:lotka}
    \begin{aligned}
      \frac{\dx}{\dt} &= \alpha x - \beta xy \,, \\
      \frac{\dy}{\dt} &= \delta xy - \gamma y \,,
    \end{aligned}
  \end{equation}
  where
  \begin{itemize}
  \item $x$ is the number of prey;
  \item $y$ is the number of predator;
  \item $\alpha, \beta, \gamma, \delta$ describe the interaction of
    the predator and prey.
  \end{itemize}
  We assume that we can observe the predators (i.e.\ $y$) but not the
  number of prey. To this end, we will create artificial values for
  the predators and preys by solving the Lotka Volterra equations with
  some initial values. We then ``observe'' the number of predator by
  perturbing the exact computed values at discrete points in time by
  additive Gaussian noise. These observations are then used as inputs
  for the particle filter.

  Formally, this model is defined by the following equations
  \begin{equation}
    \label{eq:lv}
    \begin{aligned}
      p(\bm{x}_0) &= \mathcal{N}(\bm{\mu}_0, \bm{I})\,, \\
      p(\bm{x}_t \mid \bm{x}_{t-1}) &= \mathcal{N}(\bm{x}_t;\, M(\bm{x}_{t-1}; t), \bm{I})\,, \\
      p(y_t \mid \bm{x}_t) &= \mathcal{N}(y_t;\, F(\bm{x}_{t}), 2) \,,
    \end{aligned}
  \end{equation}
  where $\bm{\mu}_0 = {(4, 6)}^T$, $\bm{I}$ denotes the $2 \times 2$
  identity matrix, $M(x_{t-1}; t)$ is the solution of the Lotka
  Volterra equations from time $t-1$ to $t$ with initial value
  $x_{t-1}$ and $F(\bm{x})$ extracts the value of the predator from
  the vector $\bm{x} = {(x_1, x_2)}^T \in \R^2$, \ie,
  \begin{equation}
    \label{eq:lv1:obs}
    F(\bm{x}) = F({(x_1, x_2)}^T) = x_2
  \end{equation}
  or equivalently
  \begin{equation}
    \label{eq:lv1:obs2}
    F(\bm{x}) = \bm{H} \bm{x}\,,
  \end{equation}
  where $\bm{H} = (0, 1) \in \R^{1 \times 2}$ is an \emph{observation
    matrix}. Additionally, we assume the values of
  $\alpha, \beta, \gamma, \delta$ are known to be
  \[
    \alpha = \gamma = 1 \qquad \text{and} \qquad \beta = \delta =
    0.1\,.
  \]

  To define a model class, we proceed as in the previous
  example. Since the particles take values in $\R^2$ we cannot use
  \texttt{double} values. Since this example should also demonstrate
  how third-party libraries can be used, instead of using
  \texttt{std::vector} or \texttt{std::array} to model the 2D vectors
  we use a vector class from the linear algebra library
  \emph{Armadillo}~\cite{armadillo}. The model can then be defined as
\begin{minted}{cpp}
class LotkaVolterra : public smcpf::Model<arma::dvec, double, double> {
\end{minted}
  where the first parameter \texttt{arma::dvec} defines the particle
  type (\ie, the two dimensional vector holding the value of the prey
  and predator, respectively), the second parameter is again the
  observation type and the additional \texttt{double} parameter
  denotes the type of the time steps.

  As above, we need to override the four pure virtual methods of the
  \texttt{Model} class. Here, a zero particle is the zero vector, \ie,
\begin{minted}{cpp}
  arma::dvec zero_particle() override { return {0, 0}; }
\end{minted}
  To sample from the prior we use the functor
  \texttt{BivariateGaussian} that can be found in the
  \texttt{functors.hh} file in the \texttt{apps/example2} folder. This
  functor represents a two-dimensional Gaussian
  distribution. Internally it uses a library called
  \emph{StatsLib}~\cite{stats} which provides a wide variety of
  statistical distributions and helper functions, in particular it
  provides classes and methods to sample from and evaluate
  multivariate normal distributions. After setting up the prior in the
  constructor of the \texttt{LotkaVolterra} class
\begin{minted}{cpp}
    const arma::dvec mu{{4}, {6}};
    const arma::dmat sigma{{1, 0}, {0, 1}};
    m_prior = BivariateNormal(mu, sigma);
\end{minted}
  we can use it to define the \texttt{sample\_prior} method
\begin{minted}{cpp}
  virtual void sample_prior(smcpf::Particle<arma::dvec> &t_particle) override {
    t_particle.set_value(m_prior());
  };
\end{minted}
  As in the previous example, to define the remaining two methods we
  have to choose a proposal density. In this example, we again
  implement a bootstrap filter, \ie, we use
  $p(\bm{x}_t \mid \bm{x}_{t-1})$ as the proposal and obtain
\begin{minted}{cpp}
  virtual double update_weight(
      const smcpf::Particle<arma::dvec> & /*t_particle_before_sampling*/,
      const smcpf::Particle<arma::dvec> &t_particle_after_sampling,
      const double &t_observation, double /*t_time*/) override {
    return stats::dnorm(t_observation,
                        extract_predator(t_particle_after_sampling), 2);
  }

  arma::dvec sample_proposal(const smcpf::Particle<arma::dvec> &t_curr_particle,
                             const double & /* t_observation */,
                             double t_time) override {
    const auto mu = evolve(t_curr_particle, t_time);
    const arma::dmat cov = {{1, 0}, {0, 1}};

    return BivariateNormal(mu, cov)();
  }
\end{minted}
  The method \texttt{stats::dnorm} from the \emph{StatsLib}
  library~\cite{stats} used in the \texttt{weight\_update} method
  provides the density function of a one-dimensional Gaussian. The
  method \texttt{extract\_predator} implements the observation
  function $F(\bm{x})$ given in~\eqref{eq:lv1:obs}.

  Inside the \texttt{sample\_proposal} method the particle is evolved
  using the model's method \texttt{evolve} that simply solves the
  Lotka Volterra equations at the current time step using the
  \texttt{odeint} methods from the C++ library Boost~\cite{boost}.

  The remaining methods of the class are again helper functions to
  process the previously generated observations. Setting up and
  running the particle filter is then done similarly as in the first
  example. Figure~\ref{fig:ex2} shows the exact values of the
  predators and preys, respectively, and the observations that were
  given to the particle filter. The two dotted lines show the sample
  means of the filtered posterior, \ie, the simulated values for the
  predators and preys.

\end{example}

The previous two examples both implement a bootstrap filter. The next
example is based on the same model as Example~\ref{ex:lv1} but uses a
different proposal.

\begin{example}[Lotka Volterra using optimal proposal]\label{ex:lv2}
  Consider the model from the previous example defined
  in~\eqref{eq:lv}. In~\cite{Doucet2000} Doucet, Godsill and Andrieu
  showed that for models of this form an optimal proposal density
  exists in the sense that the variance of the importance weights is
  minimised.

  In general, this proposal is given by
  \begin{align*}
    q(x_k \mid x^{(i)}_{k-1}, y_k) &= p(x_k \mid x_k^{(i)}, z_k)
  \end{align*}
  Substituting this into~\eqref{eq:weight_update} yields
  \[
    w_k^{(i)} \propto w_{k-1}^{(i)} p(y_k \mid x_{k-1}^{(i)}).
  \]
  As is illustrated in~\cite[Example 3]{Doucet2000} for models with
  nonlinear internal dynamics and linear measurements both
  $p(x_k \mid x_k^{(i)}, z_k)$ and $p(y_k \mid x_{k-1}^{(i)})$ are
  Gaussian. Our model obviously fulfils these requirements and we can
  define $p(x_k \mid x_k^{(i)}, z_k)$ and $p(y_k \mid x_{k-1}^{(i)})$
  as follows (see~\cite{Doucet2000} for the derivation). First let
  \begin{align*}
    \bm{\Sigma}_t^{-1} &= {(\bm{I} + 1/2 \bm{H}^T \bm{H})}^{-1}\,\\
    \bm{\mu}_t &= \bm{\Sigma}_t (M(\bm{x}_{t-1}; t) + 1/2 H^T y_t)\,.
  \end{align*}
  We can define
  \begin{equation}
    \label{eq:optimal_proposal}
    p(\bm{x}_t \mid \bm{x}_{t-1}, y_t) = \mathcal{N}(\bm{x}_t;\, \bm{\mu}_t, \bm{\Sigma}_t)
  \end{equation}
  and
  \begin{equation}
    \label{eq:optimal_weight}
    p(y_t \mid \bm{x}_{t-1}) = \mathcal{N}(y_t; \, \bm{H} M(\bm{x}_{t-1}; t), \,  \bm{H} \bm{H}^T + 2)\,.
  \end{equation}
  Since this choice of proposal density minimises the variance of the
  weights which implies an increase of the effective sampling size
  compared to the previous approach, we expect resampling to be
  performed less often. This can be easily checked using the output
  files that the examples generate, since the history also saves
  whether resampling has been performed at a particular time step. And
  indeed, while the previous example requires resampling to be
  performed at around 20 steps (of a total of 50 steps), using the
  optimal proposal this number is reduced to five resampling steps
  (both exemplar runs used 5000 particles and a resampling threshold
  of 50\%). This also has an effect on the speed of the particle
  filter since resampling is the only step that is not easily
  parallelized. Therefore, one wants to resample as few times as
  possible.

  The code for this example can be found in the folder
  \texttt{apps/example3} but is essentially the same as in the
  previous example with the \texttt{update\_weight} and
  \texttt{sample\_proposal} methods altered to
  implement~\eqref{eq:optimal_weight} and~\eqref{eq:optimal_proposal}.
  For completeness, the results using this example are plotted in
  Figure~\ref{fig:ex3}.
\end{example}

\begin{landscape}
  \begin{figure}
    \centering
    \includegraphics[width=\linewidth,height=\textheight,keepaspectratio]{figures/figure_ex1.pdf}
    \caption{Results from one exemplar run using the model and
      particle filter described in Example~\ref{ex:1}. The black line
      with crosses shows the exact values, the dotted line shows the
      generated observations and the red line (the line without
      points) is the result using 400 particles, systematic resampling
      and a resampling threshold of 50\%.}%
    \label{fig:ex1}
  \end{figure}
\end{landscape}

\begin{landscape}
  \begin{figure}
    \centering
    \includegraphics[width=\linewidth,height=\textheight,keepaspectratio]{figures/figure_ex2.pdf}
    \caption{Results from one exemplar run using the model and
      particle filter described in Example~\ref{ex:lv1}. Here, 1000
      particles and the default values of the particle filter were
      used.}%
    \label{fig:ex2}
  \end{figure}
\end{landscape}

\begin{landscape}
  \begin{figure}
    \centering
    \includegraphics[width=\linewidth,height=\textheight,keepaspectratio]{figures/figure_ex3.pdf}
    \caption{Results from one exemplar run using the model and
      particle filter described in Example~\ref{ex:lv2}. Here, 1000
      particles and the default values of the particle filter were
      used.}%
    \label{fig:ex3}
  \end{figure}
\end{landscape}


%%%Local Variables:
%%% TeX-command-extra-options: "-shell-escape"
%%% mode: latex
%%% TeX-master: "../main"
%%% End:
 















































