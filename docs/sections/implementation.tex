\section*{Implementation}
In this section we discuss an implemenation of a particle filter in
C++. The particle filter itself is implemented as a dependency-free
header-only generic library that, while being easy to set up and use,
is versatile and can be used with a wide variety of problems. This is
demonstrated by three examples, two of which are based on the same
problem but are using different prior and proposal distributions.

The library consists of the following classes:
\begin{itemize}
\item \texttt{Particle}
\item \texttt{ParticleFilter}
\item \texttt{Model}
\item \texttt{History}
\end{itemize}
All of these classes are templated to allow for arbitrary particle
types, e.\,g.\ the \texttt{Particle} class is of the following form
\begin{minted}{cpp}
template <class PT> 
class Particle {
private:
  PT m_value;
  double m_weight;
...
}; 
\end{minted}
where the particle type \texttt{PT} could take values of some finite
set, be a real number (\ie a \texttt{double}) or a $n$-dimensional
vector etc. The \texttt{ParticleFilter} class implements the
algorithms introduced above. It takes the following template
parameters
\begin{minted}{cpp}
template <class PT, class OT, size_t N, 
          bool enable_history = false,
          bool parallel = false>
class ParticleFilter
\end{minted}
where \texttt{PT} and \texttt{OT} denote the type of particle and
observation, respectively, \texttt{N} is the number of particles,
\texttt{enable\_history} specifies whether all particles from every
timestep should be held in memory (e.\,g.\ for debugging or plotting)
and \texttt{parallel} specifies whether certain functions should run
in parallel (see below). Apart from these compile-time parameters, to
construct a \texttt{ParticleFilter} one also needs to provide an
instance of a \texttt{Model}, a resampling strategy\footnote{At the
  moment, only systematic resampling is implemented.}, a resampling
treshhold and an initial seed for the random number generator
(rng). Only the first parameter is mandatory, i.\,e.\ the signature of
the constructor of the \texttt{ParticleFilter} class is given by
\begin{minted}{cpp}
ParticleFilter(
      Model<PT, OT> *t_model,
      double t_seed = 0,
      ResamplingStrategy t_strategy = ResamplingStrategy::RESAMPLING_SYSTEMATIC,
      double t_treshhold = 0.5)
\end{minted}

The \texttt{Model} class is implemented as an abstract base class
(sometimes called interface), meaning that the class itself cannot be
instaniated. Therefore, in order to define a model, a class that is
derived from \texttt{Model} has to be implemented. The \texttt{Model}
class defines the following four pure virtual functions that every
model that derives from this class needs to override.
\begin{minted}{cpp}
virtual void sample_prior(Particle<PT> &t_particle) = 0;

virtual PT zero_particle() = 0;

virtual double update_weight(const Particle<PT> &t_particle_before_sampling,
			     const Particle<PT> &t_particle_after_sampling,
                             const OT &t_observation,
                             double t_time) = 0;

virtual PT sample_proposal(const Particle<PT> &t_particle,
                           const OT &t_observation,
                           double t_time) = 0;
\end{minted}

All these methods are automatically called by the
\texttt{ParticleFilter}. The method \texttt{sample\_prior} is used to
initialise the set of particles. It has to set the value of the given
particle \texttt{t\_particle} using the \texttt{set\_value} method of
the particle. This is different from the \texttt{update\_weight} and
\texttt{sample\_proposal} methods that \emph{do not} alter the
particle themselves. Rather they should return the updated value of
the particle's weight and value, respectively (see examples below).

The first paramter of the \texttt{update\_weight} method is the
particle before \texttt{sample\_proposal} is called and the second
after it is called. This is useful, since the developer cannot specify
the order in which these two methods are called. However, some models
require the value of the particle before it has been updated (cf.\
Example~\ref{ex:lv1}) and some after the sampling step (cf.\
Example~\ref{ex:lv2}).

\begin{example}
  Consider the following model (this example has been studied in a
  number of publications before, see for
  example~\cite{arulampalam,gordon,kitagawa}). Let
  \begin{gather*}
    f(x_n \mid x_{n-1}) = \mathcal{N}(x_n;\, h_n(x_{n-1}), \sigma_{\text{sys}})\\
    g(y_n \mid x_n) = \mathcal{N}(y_n;\, \frac{{x_n}^2}{20},
    \sigma_{\text{obs}})
  \end{gather*}
  where
  \[
    h_n(x_{n-1}) = \frac{1}{2}x_{n-1} + \frac{25 x_{n-1}}{1 + x_{n-1}^2} + 8 \cos(1.2n) \\
  \]
\end{example}

\begin{example}[Lotka Volterra using bootstrap filter]\label{ex:lv1}
\end{example}

\begin{example}[Lotka Volterra using optimal proposal]\label{ex:lv2}
  
\end{example}

%%%Local Variables:
%%% TeX-command-extra-options: "-shell-escape"
%%% mode: latex
%%% TeX-master: "../main"
%%% End:
